%% Generated by Sphinx.
\def\sphinxdocclass{report}
\documentclass[letterpaper,10pt,brazil]{sphinxmanual}
\ifdefined\pdfpxdimen
   \let\sphinxpxdimen\pdfpxdimen\else\newdimen\sphinxpxdimen
\fi \sphinxpxdimen=.75bp\relax

\PassOptionsToPackage{warn}{textcomp}
\usepackage[utf8]{inputenc}
\ifdefined\DeclareUnicodeCharacter
% support both utf8 and utf8x syntaxes
  \ifdefined\DeclareUnicodeCharacterAsOptional
    \def\sphinxDUC#1{\DeclareUnicodeCharacter{"#1}}
  \else
    \let\sphinxDUC\DeclareUnicodeCharacter
  \fi
  \sphinxDUC{00A0}{\nobreakspace}
  \sphinxDUC{2500}{\sphinxunichar{2500}}
  \sphinxDUC{2502}{\sphinxunichar{2502}}
  \sphinxDUC{2514}{\sphinxunichar{2514}}
  \sphinxDUC{251C}{\sphinxunichar{251C}}
  \sphinxDUC{2572}{\textbackslash}
\fi
\usepackage{cmap}
\usepackage[T1]{fontenc}
\usepackage{amsmath,amssymb,amstext}
\usepackage{babel}



\usepackage{times}
\expandafter\ifx\csname T@LGR\endcsname\relax
\else
% LGR was declared as font encoding
  \substitutefont{LGR}{\rmdefault}{cmr}
  \substitutefont{LGR}{\sfdefault}{cmss}
  \substitutefont{LGR}{\ttdefault}{cmtt}
\fi
\expandafter\ifx\csname T@X2\endcsname\relax
  \expandafter\ifx\csname T@T2A\endcsname\relax
  \else
  % T2A was declared as font encoding
    \substitutefont{T2A}{\rmdefault}{cmr}
    \substitutefont{T2A}{\sfdefault}{cmss}
    \substitutefont{T2A}{\ttdefault}{cmtt}
  \fi
\else
% X2 was declared as font encoding
  \substitutefont{X2}{\rmdefault}{cmr}
  \substitutefont{X2}{\sfdefault}{cmss}
  \substitutefont{X2}{\ttdefault}{cmtt}
\fi


\usepackage[Sonny]{fncychap}
\ChNameVar{\Large\normalfont\sffamily}
\ChTitleVar{\Large\normalfont\sffamily}
\usepackage{sphinx}

\fvset{fontsize=\small}
\usepackage{geometry}


% Include hyperref last.
\usepackage{hyperref}
% Fix anchor placement for figures with captions.
\usepackage{hypcap}% it must be loaded after hyperref.
% Set up styles of URL: it should be placed after hyperref.
\urlstyle{same}

\addto\captionsbrazil{\renewcommand{\contentsname}{Contents:}}

\usepackage{sphinxmessages}
\setcounter{tocdepth}{1}



\title{LP\sphinxhyphen{}A2}
\date{Dec 08, 2020}
\release{1.0.0}
\author{Erick Brito, Germano Andrade, João Alcindo, Patrick Saul, Sávio Vinicius}
\newcommand{\sphinxlogo}{\vbox{}}
\renewcommand{\releasename}{Release}
\makeindex
\begin{document}

\ifdefined\shorthandoff
  \ifnum\catcode`\=\string=\active\shorthandoff{=}\fi
  \ifnum\catcode`\"=\active\shorthandoff{"}\fi
\fi

\pagestyle{empty}
\sphinxmaketitle
\pagestyle{plain}
\sphinxtableofcontents
\pagestyle{normal}
\phantomsection\label{\detokenize{index::doc}}



\chapter{modules}
\label{\detokenize{modules:modules}}\label{\detokenize{modules::doc}}

\section{classe\_Fifa\_limp module}
\label{\detokenize{classe_Fifa_limp:module-classe_Fifa_limp}}\label{\detokenize{classe_Fifa_limp:classe-fifa-limp-module}}\label{\detokenize{classe_Fifa_limp::doc}}\index{module@\spxentry{module}!classe\_Fifa\_limp@\spxentry{classe\_Fifa\_limp}}\index{classe\_Fifa\_limp@\spxentry{classe\_Fifa\_limp}!module@\spxentry{module}}\index{Fifa\_limp (class in classe\_Fifa\_limp)@\spxentry{Fifa\_limp}\spxextra{class in classe\_Fifa\_limp}}

\begin{fulllineitems}
\phantomsection\label{\detokenize{classe_Fifa_limp:classe_Fifa_limp.Fifa_limp}}\pysiglinewithargsret{\sphinxbfcode{\sphinxupquote{class }}\sphinxcode{\sphinxupquote{classe\_Fifa\_limp.}}\sphinxbfcode{\sphinxupquote{Fifa\_limp}}}{\emph{\DUrole{n}{dataframe}}}{}
Bases: \sphinxcode{\sphinxupquote{object}}

Classe responsável pelas funções que farão a limpeza do datrafame sujo.
\index{dataframe (classe\_Fifa\_limp.Fifa\_limp attribute)@\spxentry{dataframe}\spxextra{classe\_Fifa\_limp.Fifa\_limp attribute}}

\begin{fulllineitems}
\phantomsection\label{\detokenize{classe_Fifa_limp:classe_Fifa_limp.Fifa_limp.dataframe}}\pysigline{\sphinxbfcode{\sphinxupquote{dataframe}}}
dataframe.
\begin{quote}\begin{description}
\item[{Type}] \leavevmode
pandas.core.frame.dataframe

\end{description}\end{quote}

\end{fulllineitems}

\index{trocar\_pes\_para\_cm() (classe\_Fifa\_limp.Fifa\_limp method)@\spxentry{trocar\_pes\_para\_cm()}\spxextra{classe\_Fifa\_limp.Fifa\_limp method}}

\begin{fulllineitems}
\phantomsection\label{\detokenize{classe_Fifa_limp:classe_Fifa_limp.Fifa_limp.trocar_pes_para_cm}}\pysiglinewithargsret{\sphinxbfcode{\sphinxupquote{trocar\_pes\_para\_cm}}}{}{}
Transforma uma string com a altura medida em pés em um float com essa altura convertida para centímetros.

\end{fulllineitems}

\index{trocar\_lbs\_para\_kg() (classe\_Fifa\_limp.Fifa\_limp method)@\spxentry{trocar\_lbs\_para\_kg()}\spxextra{classe\_Fifa\_limp.Fifa\_limp method}}

\begin{fulllineitems}
\phantomsection\label{\detokenize{classe_Fifa_limp:classe_Fifa_limp.Fifa_limp.trocar_lbs_para_kg}}\pysiglinewithargsret{\sphinxbfcode{\sphinxupquote{trocar\_lbs\_para\_kg}}}{}{}
Transforma uma string com o peso, em libras, em um float com esse peso convertido para quilogramas.

\end{fulllineitems}

\index{trocar\_valores\_str\_p\_int() (classe\_Fifa\_limp.Fifa\_limp method)@\spxentry{trocar\_valores\_str\_p\_int()}\spxextra{classe\_Fifa\_limp.Fifa\_limp method}}

\begin{fulllineitems}
\phantomsection\label{\detokenize{classe_Fifa_limp:classe_Fifa_limp.Fifa_limp.trocar_valores_str_p_int}}\pysiglinewithargsret{\sphinxbfcode{\sphinxupquote{trocar\_valores\_str\_p\_int}}}{}{}
Transforma uma string contendo o valor, na forma MK(M = Milhões, K = Milhares), em um float com o valor real.

\end{fulllineitems}

\index{traduzir\_posicoes() (classe\_Fifa\_limp.Fifa\_limp method)@\spxentry{traduzir\_posicoes()}\spxextra{classe\_Fifa\_limp.Fifa\_limp method}}

\begin{fulllineitems}
\phantomsection\label{\detokenize{classe_Fifa_limp:classe_Fifa_limp.Fifa_limp.traduzir_posicoes}}\pysiglinewithargsret{\sphinxbfcode{\sphinxupquote{traduzir\_posicoes}}}{}{}
Traduz a string com a sigla da posição em uma string com nome da posição.

\end{fulllineitems}

\index{colunas\_desejadas() (classe\_Fifa\_limp.Fifa\_limp method)@\spxentry{colunas\_desejadas()}\spxextra{classe\_Fifa\_limp.Fifa\_limp method}}

\begin{fulllineitems}
\phantomsection\label{\detokenize{classe_Fifa_limp:classe_Fifa_limp.Fifa_limp.colunas_desejadas}}\pysiglinewithargsret{\sphinxbfcode{\sphinxupquote{colunas\_desejadas}}}{}{}
Seleciona uma lista de colunas do dataframe.

\end{fulllineitems}

\index{porcentagem\_overall\_potential() (classe\_Fifa\_limp.Fifa\_limp method)@\spxentry{porcentagem\_overall\_potential()}\spxextra{classe\_Fifa\_limp.Fifa\_limp method}}

\begin{fulllineitems}
\phantomsection\label{\detokenize{classe_Fifa_limp:classe_Fifa_limp.Fifa_limp.porcentagem_overall_potential}}\pysiglinewithargsret{\sphinxbfcode{\sphinxupquote{porcentagem\_overall\_potential}}}{}{}
Adiciona uma nova coluna que calcula a porcentagem da diferença de potencial e overall.

\end{fulllineitems}

\index{escrever\_csv() (classe\_Fifa\_limp.Fifa\_limp method)@\spxentry{escrever\_csv()}\spxextra{classe\_Fifa\_limp.Fifa\_limp method}}

\begin{fulllineitems}
\phantomsection\label{\detokenize{classe_Fifa_limp:classe_Fifa_limp.Fifa_limp.escrever_csv}}\pysiglinewithargsret{\sphinxbfcode{\sphinxupquote{escrever\_csv}}}{}{}
Escreve o dataframe em um csv.

\end{fulllineitems}

\index{arredondar\_valores() (classe\_Fifa\_limp.Fifa\_limp method)@\spxentry{arredondar\_valores()}\spxextra{classe\_Fifa\_limp.Fifa\_limp method}}

\begin{fulllineitems}
\phantomsection\label{\detokenize{classe_Fifa_limp:classe_Fifa_limp.Fifa_limp.arredondar_valores}}\pysiglinewithargsret{\sphinxbfcode{\sphinxupquote{arredondar\_valores}}}{}{}
Arredonda valores.

\end{fulllineitems}

\index{cria\_skills() (classe\_Fifa\_limp.Fifa\_limp method)@\spxentry{cria\_skills()}\spxextra{classe\_Fifa\_limp.Fifa\_limp method}}

\begin{fulllineitems}
\phantomsection\label{\detokenize{classe_Fifa_limp:classe_Fifa_limp.Fifa_limp.cria_skills}}\pysiglinewithargsret{\sphinxbfcode{\sphinxupquote{cria\_skills}}}{}{}
Cria habilidades

\end{fulllineitems}

\index{arredondar\_valores() (classe\_Fifa\_limp.Fifa\_limp method)@\spxentry{arredondar\_valores()}\spxextra{classe\_Fifa\_limp.Fifa\_limp method}}

\begin{fulllineitems}
\phantomsection\label{\detokenize{classe_Fifa_limp:id0}}\pysiglinewithargsret{\sphinxbfcode{\sphinxupquote{arredondar\_valores}}}{\emph{\DUrole{n}{column}}, \emph{\DUrole{n}{precisao}}}{}
Arredonda valores da coluna desejada com a precisão desejada.
\begin{quote}\begin{description}
\item[{Parameters}] \leavevmode\begin{itemize}
\item {} 
\sphinxstyleliteralstrong{\sphinxupquote{column}} (\sphinxstyleliteralemphasis{\sphinxupquote{pandas.core.series.Series}}) \textendash{} Coluna do dataframe.

\item {} 
\sphinxstyleliteralstrong{\sphinxupquote{precisao}} (\sphinxstyleliteralemphasis{\sphinxupquote{int}}) \textendash{} Quantas casas decimais se quer arredondar.

\end{itemize}

\end{description}\end{quote}

\end{fulllineitems}

\index{colunas\_desejadas() (classe\_Fifa\_limp.Fifa\_limp method)@\spxentry{colunas\_desejadas()}\spxextra{classe\_Fifa\_limp.Fifa\_limp method}}

\begin{fulllineitems}
\phantomsection\label{\detokenize{classe_Fifa_limp:id1}}\pysiglinewithargsret{\sphinxbfcode{\sphinxupquote{colunas\_desejadas}}}{\emph{\DUrole{n}{list\_colunas}}}{}
Seleciona uma lista de colunas desejadas do dataframe.
\begin{quote}\begin{description}
\item[{Parameters}] \leavevmode
\sphinxstyleliteralstrong{\sphinxupquote{list\_colunas}} (\sphinxstyleliteralemphasis{\sphinxupquote{list}}) \textendash{} Lista com as colunas desejadas.

\end{description}\end{quote}

\end{fulllineitems}

\index{cria\_skills() (classe\_Fifa\_limp.Fifa\_limp method)@\spxentry{cria\_skills()}\spxextra{classe\_Fifa\_limp.Fifa\_limp method}}

\begin{fulllineitems}
\phantomsection\label{\detokenize{classe_Fifa_limp:id2}}\pysiglinewithargsret{\sphinxbfcode{\sphinxupquote{cria\_skills}}}{}{}
Cria os atributos técnicos dos jogadores baseando\sphinxhyphen{}se nas habilidades de cada um.

\end{fulllineitems}

\index{escrever\_csv() (classe\_Fifa\_limp.Fifa\_limp method)@\spxentry{escrever\_csv()}\spxextra{classe\_Fifa\_limp.Fifa\_limp method}}

\begin{fulllineitems}
\phantomsection\label{\detokenize{classe_Fifa_limp:id3}}\pysiglinewithargsret{\sphinxbfcode{\sphinxupquote{escrever\_csv}}}{\emph{\DUrole{n}{caminho\_arquivo}}}{}
Escreve o dataframe em um arquivo csv.
\begin{quote}\begin{description}
\item[{Parameters}] \leavevmode
\sphinxstyleliteralstrong{\sphinxupquote{caminho\_arquivo}} (\sphinxstyleliteralemphasis{\sphinxupquote{str}}) \textendash{} Path do arquivo.

\end{description}\end{quote}

\end{fulllineitems}

\index{porcentagem\_overall\_potential() (classe\_Fifa\_limp.Fifa\_limp method)@\spxentry{porcentagem\_overall\_potential()}\spxextra{classe\_Fifa\_limp.Fifa\_limp method}}

\begin{fulllineitems}
\phantomsection\label{\detokenize{classe_Fifa_limp:id4}}\pysiglinewithargsret{\sphinxbfcode{\sphinxupquote{porcentagem\_overall\_potential}}}{\emph{\DUrole{n}{potential}}, \emph{\DUrole{n}{overall}}, \emph{\DUrole{n}{new\_column}}}{}
Adiciona uma nova coluna que calcula a porcentagem da diferença de potencial e overall.
\begin{quote}\begin{description}
\item[{Parameters}] \leavevmode\begin{itemize}
\item {} 
\sphinxstyleliteralstrong{\sphinxupquote{potential}} (\sphinxstyleliteralemphasis{\sphinxupquote{str}}) \textendash{} String com o nome da coluna.

\item {} 
\sphinxstyleliteralstrong{\sphinxupquote{overall}} (\sphinxstyleliteralemphasis{\sphinxupquote{str}}) \textendash{} String com o nome da coluna.

\item {} 
\sphinxstyleliteralstrong{\sphinxupquote{new\_column}} (\sphinxstyleliteralemphasis{\sphinxupquote{str}}) \textendash{} String com o nome da nova coluna.

\end{itemize}

\end{description}\end{quote}

\end{fulllineitems}

\index{traduzir\_posicoes() (classe\_Fifa\_limp.Fifa\_limp method)@\spxentry{traduzir\_posicoes()}\spxextra{classe\_Fifa\_limp.Fifa\_limp method}}

\begin{fulllineitems}
\phantomsection\label{\detokenize{classe_Fifa_limp:id5}}\pysiglinewithargsret{\sphinxbfcode{\sphinxupquote{traduzir\_posicoes}}}{\emph{\DUrole{n}{column}}}{}
Aplica a função traduzir\_posicoes() \sphinxhyphen{}\textgreater{} módulo {[}limpeza\_dados\_fifa.py{]}\sphinxhyphen{} na coluna especificada.
\begin{quote}\begin{description}
\item[{Parameters}] \leavevmode
\sphinxstyleliteralstrong{\sphinxupquote{column}} (\sphinxstyleliteralemphasis{\sphinxupquote{pandas.core.series.Series}}) \textendash{} Coluna do dataframe.

\end{description}\end{quote}

\end{fulllineitems}

\index{trocar\_lbs\_para\_kg() (classe\_Fifa\_limp.Fifa\_limp method)@\spxentry{trocar\_lbs\_para\_kg()}\spxextra{classe\_Fifa\_limp.Fifa\_limp method}}

\begin{fulllineitems}
\phantomsection\label{\detokenize{classe_Fifa_limp:id6}}\pysiglinewithargsret{\sphinxbfcode{\sphinxupquote{trocar\_lbs\_para\_kg}}}{\emph{\DUrole{n}{column}}}{}
Aplica a função trocar\_lbs\_para\_kg() \sphinxhyphen{}\textgreater{} módulo {[}limpeza\_dados\_fifa.py{]}\sphinxhyphen{} na coluna especificada.
\begin{quote}\begin{description}
\item[{Parameters}] \leavevmode
\sphinxstyleliteralstrong{\sphinxupquote{column}} (\sphinxstyleliteralemphasis{\sphinxupquote{pandas.core.series.Series}}) \textendash{} Coluna do dataframe.

\end{description}\end{quote}

\end{fulllineitems}

\index{trocar\_pes\_para\_cm() (classe\_Fifa\_limp.Fifa\_limp method)@\spxentry{trocar\_pes\_para\_cm()}\spxextra{classe\_Fifa\_limp.Fifa\_limp method}}

\begin{fulllineitems}
\phantomsection\label{\detokenize{classe_Fifa_limp:id7}}\pysiglinewithargsret{\sphinxbfcode{\sphinxupquote{trocar\_pes\_para\_cm}}}{\emph{\DUrole{n}{column}}}{}
Aplica a função trocar\_pes\_para\_cm() \sphinxhyphen{}\textgreater{} módulo {[}limpeza\_dados\_fifa.py{]}\sphinxhyphen{} na coluna especificada.
\begin{quote}\begin{description}
\item[{Parameters}] \leavevmode
\sphinxstyleliteralstrong{\sphinxupquote{column}} (\sphinxstyleliteralemphasis{\sphinxupquote{pandas.core.series.Series}}) \textendash{} Coluna do dataframe.

\end{description}\end{quote}

\end{fulllineitems}

\index{trocar\_valores\_str\_p\_int() (classe\_Fifa\_limp.Fifa\_limp method)@\spxentry{trocar\_valores\_str\_p\_int()}\spxextra{classe\_Fifa\_limp.Fifa\_limp method}}

\begin{fulllineitems}
\phantomsection\label{\detokenize{classe_Fifa_limp:id8}}\pysiglinewithargsret{\sphinxbfcode{\sphinxupquote{trocar\_valores\_str\_p\_int}}}{\emph{\DUrole{n}{column}}}{}
Aplica a função trocar\_valores\_str\_p\_int() \sphinxhyphen{}\textgreater{} módulo {[}limpeza\_dados\_fifa.py{]}\sphinxhyphen{} na coluna especificada.
\begin{quote}\begin{description}
\item[{Parameters}] \leavevmode
\sphinxstyleliteralstrong{\sphinxupquote{column}} (\sphinxstyleliteralemphasis{\sphinxupquote{pandas.core.series.Series}}) \textendash{} Coluna do dataframe.

\end{description}\end{quote}

\end{fulllineitems}


\end{fulllineitems}



\section{classe\_RS\_limp module}
\label{\detokenize{classe_RS_limp:module-classe_RS_limp}}\label{\detokenize{classe_RS_limp:classe-rs-limp-module}}\label{\detokenize{classe_RS_limp::doc}}\index{module@\spxentry{module}!classe\_RS\_limp@\spxentry{classe\_RS\_limp}}\index{classe\_RS\_limp@\spxentry{classe\_RS\_limp}!module@\spxentry{module}}\index{RS\_limp (class in classe\_RS\_limp)@\spxentry{RS\_limp}\spxextra{class in classe\_RS\_limp}}

\begin{fulllineitems}
\phantomsection\label{\detokenize{classe_RS_limp:classe_RS_limp.RS_limp}}\pysiglinewithargsret{\sphinxbfcode{\sphinxupquote{class }}\sphinxcode{\sphinxupquote{classe\_RS\_limp.}}\sphinxbfcode{\sphinxupquote{RS\_limp}}}{\emph{\DUrole{n}{dataframe}}}{}
Bases: \sphinxcode{\sphinxupquote{object}}

Classe responsável pelas funções que farão a limpeza do datrafame sujo.
\index{dataframe (classe\_RS\_limp.RS\_limp attribute)@\spxentry{dataframe}\spxextra{classe\_RS\_limp.RS\_limp attribute}}

\begin{fulllineitems}
\phantomsection\label{\detokenize{classe_RS_limp:classe_RS_limp.RS_limp.dataframe}}\pysigline{\sphinxbfcode{\sphinxupquote{dataframe}}}
Dataframe.
\begin{quote}\begin{description}
\item[{Type}] \leavevmode
pandas.core.frame.dataframe

\end{description}\end{quote}

\end{fulllineitems}

\index{trocar\_bools\_para\_int() (classe\_RS\_limp.RS\_limp method)@\spxentry{trocar\_bools\_para\_int()}\spxextra{classe\_RS\_limp.RS\_limp method}}

\begin{fulllineitems}
\phantomsection\label{\detokenize{classe_RS_limp:classe_RS_limp.RS_limp.trocar_bools_para_int}}\pysiglinewithargsret{\sphinxbfcode{\sphinxupquote{trocar\_bools\_para\_int}}}{}{}
Transforma os booleanos em inteiros.

\end{fulllineitems}

\index{drop\_na() (classe\_RS\_limp.RS\_limp method)@\spxentry{drop\_na()}\spxextra{classe\_RS\_limp.RS\_limp method}}

\begin{fulllineitems}
\phantomsection\label{\detokenize{classe_RS_limp:classe_RS_limp.RS_limp.drop_na}}\pysiglinewithargsret{\sphinxbfcode{\sphinxupquote{drop\_na}}}{}{}
Faz a mesma função de np.dropna.

\end{fulllineitems}

\index{escrever\_csv() (classe\_RS\_limp.RS\_limp method)@\spxentry{escrever\_csv()}\spxextra{classe\_RS\_limp.RS\_limp method}}

\begin{fulllineitems}
\phantomsection\label{\detokenize{classe_RS_limp:classe_RS_limp.RS_limp.escrever_csv}}\pysiglinewithargsret{\sphinxbfcode{\sphinxupquote{escrever\_csv}}}{}{}
Escreve o dataframe em um csv.

\end{fulllineitems}

\index{drop\_na() (classe\_RS\_limp.RS\_limp method)@\spxentry{drop\_na()}\spxextra{classe\_RS\_limp.RS\_limp method}}

\begin{fulllineitems}
\phantomsection\label{\detokenize{classe_RS_limp:id0}}\pysiglinewithargsret{\sphinxbfcode{\sphinxupquote{drop\_na}}}{}{}
Dropa as colunas e linhas que possuem valores nulos.

\end{fulllineitems}

\index{escrever\_csv() (classe\_RS\_limp.RS\_limp method)@\spxentry{escrever\_csv()}\spxextra{classe\_RS\_limp.RS\_limp method}}

\begin{fulllineitems}
\phantomsection\label{\detokenize{classe_RS_limp:id1}}\pysiglinewithargsret{\sphinxbfcode{\sphinxupquote{escrever\_csv}}}{\emph{\DUrole{n}{caminho\_arquivo}}}{}
Escreve o dataframe em um arquivo csv.
\begin{quote}\begin{description}
\item[{Parameters}] \leavevmode\begin{itemize}
\item {} 
\sphinxstyleliteralstrong{\sphinxupquote{caminho\_arquivo}} (\sphinxstyleliteralemphasis{\sphinxupquote{str}}) \textendash{} 

\item {} 
\sphinxstyleliteralstrong{\sphinxupquote{do arquivo.}} (\sphinxstyleliteralemphasis{\sphinxupquote{Path}}) \textendash{} 

\end{itemize}

\end{description}\end{quote}

\end{fulllineitems}

\index{trocar\_bools\_para\_int() (classe\_RS\_limp.RS\_limp method)@\spxentry{trocar\_bools\_para\_int()}\spxextra{classe\_RS\_limp.RS\_limp method}}

\begin{fulllineitems}
\phantomsection\label{\detokenize{classe_RS_limp:id2}}\pysiglinewithargsret{\sphinxbfcode{\sphinxupquote{trocar\_bools\_para\_int}}}{\emph{\DUrole{n}{columns}}}{}
Transforma os booleanos em inteiros
\begin{quote}\begin{description}
\item[{Parameters}] \leavevmode
\sphinxstyleliteralstrong{\sphinxupquote{columns}} (\sphinxstyleliteralemphasis{\sphinxupquote{pandas.core.series.Series}}) \textendash{} Coluna específica do dataframe

\end{description}\end{quote}

\end{fulllineitems}


\end{fulllineitems}



\section{funcoes\_auxiliares module}
\label{\detokenize{funcoes_auxiliares:module-funcoes_auxiliares}}\label{\detokenize{funcoes_auxiliares:funcoes-auxiliares-module}}\label{\detokenize{funcoes_auxiliares::doc}}\index{module@\spxentry{module}!funcoes\_auxiliares@\spxentry{funcoes\_auxiliares}}\index{funcoes\_auxiliares@\spxentry{funcoes\_auxiliares}!module@\spxentry{module}}\index{lbs\_para\_kg() (in module funcoes\_auxiliares)@\spxentry{lbs\_para\_kg()}\spxextra{in module funcoes\_auxiliares}}

\begin{fulllineitems}
\phantomsection\label{\detokenize{funcoes_auxiliares:funcoes_auxiliares.lbs_para_kg}}\pysiglinewithargsret{\sphinxcode{\sphinxupquote{funcoes\_auxiliares.}}\sphinxbfcode{\sphinxupquote{lbs\_para\_kg}}}{\emph{\DUrole{n}{weight}}}{}
Função que transforma uma string com o peso, em libras, em um float com esse peso convertido para quilogramas.
\begin{quote}\begin{description}
\item[{Parameters}] \leavevmode
\sphinxstyleliteralstrong{\sphinxupquote{weight}} (\sphinxstyleliteralemphasis{\sphinxupquote{str}}) \textendash{} String contendo o peso, em libras, dos jogadores.

\item[{Returns}] \leavevmode
Float contendo o peso, em quilogramas, dos jogadores.

\item[{Return type}] \leavevmode
float

\end{description}\end{quote}

\end{fulllineitems}

\index{pes\_para\_cm() (in module funcoes\_auxiliares)@\spxentry{pes\_para\_cm()}\spxextra{in module funcoes\_auxiliares}}

\begin{fulllineitems}
\phantomsection\label{\detokenize{funcoes_auxiliares:funcoes_auxiliares.pes_para_cm}}\pysiglinewithargsret{\sphinxcode{\sphinxupquote{funcoes\_auxiliares.}}\sphinxbfcode{\sphinxupquote{pes\_para\_cm}}}{\emph{\DUrole{n}{height}}}{}
Função que transforma uma string com a altura medida em pés em um float com essa altura convertida para centímetros.
\begin{quote}\begin{description}
\item[{Parameters}] \leavevmode
\sphinxstyleliteralstrong{\sphinxupquote{height}} (\sphinxstyleliteralemphasis{\sphinxupquote{str}}) \textendash{} String contendo a altura, em pés, dos jogadores.

\item[{Returns}] \leavevmode
Float contendo a altura, em centímetros, dos jogadores.

\item[{Return type}] \leavevmode
float

\end{description}\end{quote}

\end{fulllineitems}

\index{preco\_str\_p\_int() (in module funcoes\_auxiliares)@\spxentry{preco\_str\_p\_int()}\spxextra{in module funcoes\_auxiliares}}

\begin{fulllineitems}
\phantomsection\label{\detokenize{funcoes_auxiliares:funcoes_auxiliares.preco_str_p_int}}\pysiglinewithargsret{\sphinxcode{\sphinxupquote{funcoes\_auxiliares.}}\sphinxbfcode{\sphinxupquote{preco\_str\_p\_int}}}{\emph{\DUrole{n}{value}}}{}
Função que transforma uma string contendo o valor, na forma MK(M = Milhões, K = Milhares), em um int com o valor real.
\begin{quote}\begin{description}
\item[{Parameters}] \leavevmode
\sphinxstyleliteralstrong{\sphinxupquote{value}} (\sphinxstyleliteralemphasis{\sphinxupquote{str}}) \textendash{} String da forma ‘\texteuro{}x(MK)’, que representa o preço x em M = Milhões ou K = Milhares de Euros.

\item[{Returns}] \leavevmode
Int com o preço real do jogador.

\item[{Return type}] \leavevmode
int

\end{description}\end{quote}

\end{fulllineitems}

\index{traduz\_posicao() (in module funcoes\_auxiliares)@\spxentry{traduz\_posicao()}\spxextra{in module funcoes\_auxiliares}}

\begin{fulllineitems}
\phantomsection\label{\detokenize{funcoes_auxiliares:funcoes_auxiliares.traduz_posicao}}\pysiglinewithargsret{\sphinxcode{\sphinxupquote{funcoes\_auxiliares.}}\sphinxbfcode{\sphinxupquote{traduz\_posicao}}}{\emph{\DUrole{n}{position}}}{}
Função que traduz a string com a sigla da posição em uma string com nome da posição.
\begin{quote}\begin{description}
\item[{Parameters}] \leavevmode
\sphinxstyleliteralstrong{\sphinxupquote{position}} (\sphinxstyleliteralemphasis{\sphinxupquote{str}}) \textendash{} String que contém a sigla que representa a posição do jogador.

\item[{Returns}] \leavevmode
String que contém a posição do jogador.

\item[{Return type}] \leavevmode
str

\end{description}\end{quote}

\end{fulllineitems}



\section{solucao\_fifa module}
\label{\detokenize{solucao_fifa:module-solucao_fifa}}\label{\detokenize{solucao_fifa:solucao-fifa-module}}\label{\detokenize{solucao_fifa::doc}}\index{module@\spxentry{module}!solucao\_fifa@\spxentry{solucao\_fifa}}\index{solucao\_fifa@\spxentry{solucao\_fifa}!module@\spxentry{module}}\index{melhor\_time\_atual() (in module solucao\_fifa)@\spxentry{melhor\_time\_atual()}\spxextra{in module solucao\_fifa}}

\begin{fulllineitems}
\phantomsection\label{\detokenize{solucao_fifa:solucao_fifa.melhor_time_atual}}\pysiglinewithargsret{\sphinxcode{\sphinxupquote{solucao\_fifa.}}\sphinxbfcode{\sphinxupquote{melhor\_time\_atual}}}{}{}
Função que retorna uma lista com o melhor time atual, de acordo com o overall e         independente do preço.
\begin{quote}\begin{description}
\item[{Returns}] \leavevmode
Lista com os jogadores do time.

\item[{Return type}] \leavevmode
list

\end{description}\end{quote}

\end{fulllineitems}

\index{melhor\_time\_futuro() (in module solucao\_fifa)@\spxentry{melhor\_time\_futuro()}\spxextra{in module solucao\_fifa}}

\begin{fulllineitems}
\phantomsection\label{\detokenize{solucao_fifa:solucao_fifa.melhor_time_futuro}}\pysiglinewithargsret{\sphinxcode{\sphinxupquote{solucao\_fifa.}}\sphinxbfcode{\sphinxupquote{melhor\_time\_futuro}}}{}{}
Função que retorna uma string com o melhor time atual, de acordo com o potencial.
\begin{quote}\begin{description}
\item[{Returns}] \leavevmode
String com os jogadores do time.

\item[{Return type}] \leavevmode
str

\end{description}\end{quote}

\end{fulllineitems}

\index{melhores() (in module solucao\_fifa)@\spxentry{melhores()}\spxextra{in module solucao\_fifa}}

\begin{fulllineitems}
\phantomsection\label{\detokenize{solucao_fifa:solucao_fifa.melhores}}\pysiglinewithargsret{\sphinxcode{\sphinxupquote{solucao\_fifa.}}\sphinxbfcode{\sphinxupquote{melhores}}}{\emph{\DUrole{n}{p}}, \emph{\DUrole{n}{df}}}{}
Função que retorna os melhores jogadores de determinada posição.
\begin{quote}\begin{description}
\item[{Parameters}] \leavevmode\begin{itemize}
\item {} 
\sphinxstyleliteralstrong{\sphinxupquote{p}} (\sphinxstyleliteralemphasis{\sphinxupquote{str}}) \textendash{} String contendo a posição desejada.

\item {} 
\sphinxstyleliteralstrong{\sphinxupquote{df}} (\sphinxstyleliteralemphasis{\sphinxupquote{pandas.core.frame.dataframe}}) \textendash{} Dataframe.

\end{itemize}

\item[{Returns}] \leavevmode
Lista com os melhores jogadores da posição desejada.

\item[{Return type}] \leavevmode
list

\end{description}\end{quote}

\end{fulllineitems}

\index{porcentagem\_canhoto() (in module solucao\_fifa)@\spxentry{porcentagem\_canhoto()}\spxextra{in module solucao\_fifa}}

\begin{fulllineitems}
\phantomsection\label{\detokenize{solucao_fifa:solucao_fifa.porcentagem_canhoto}}\pysiglinewithargsret{\sphinxcode{\sphinxupquote{solucao\_fifa.}}\sphinxbfcode{\sphinxupquote{porcentagem\_canhoto}}}{\emph{\DUrole{n}{num}}, \emph{\DUrole{n}{df}}}{}
Esta função calcula a porcentagem de canhotos entre os num jogadores mais bem avaliados.
\begin{quote}\begin{description}
\item[{Parameters}] \leavevmode\begin{itemize}
\item {} 
\sphinxstyleliteralstrong{\sphinxupquote{num}} (\sphinxstyleliteralemphasis{\sphinxupquote{int}}) \textendash{} Quantidade num de jogadores mais bem avaliados.

\item {} 
\sphinxstyleliteralstrong{\sphinxupquote{df}} (\sphinxstyleliteralemphasis{\sphinxupquote{pandas.core.frame.dataframe}}) \textendash{} Dataframe.

\end{itemize}

\item[{Returns}] \leavevmode
String com a porcentagem de jogadores canhotos entre os num melhores.

\item[{Return type}] \leavevmode
str

\end{description}\end{quote}

\end{fulllineitems}

\index{print\_melhor\_time() (in module solucao\_fifa)@\spxentry{print\_melhor\_time()}\spxextra{in module solucao\_fifa}}

\begin{fulllineitems}
\phantomsection\label{\detokenize{solucao_fifa:solucao_fifa.print_melhor_time}}\pysiglinewithargsret{\sphinxcode{\sphinxupquote{solucao\_fifa.}}\sphinxbfcode{\sphinxupquote{print\_melhor\_time}}}{\emph{\DUrole{n}{funcao\_atual\_ou\_futuro}}}{}
Função que imprirmi o melhor time.
\begin{quote}\begin{description}
\item[{Parameters}] \leavevmode
\sphinxstyleliteralstrong{\sphinxupquote{funcao\_atual\_ou\_futuro}} (\sphinxstyleliteralemphasis{\sphinxupquote{list}}) \textendash{} Lista contendo um time de jogadores

\item[{Returns}] \leavevmode
Texto organizando o time por posição

\item[{Return type}] \leavevmode
str

\end{description}\end{quote}

\end{fulllineitems}

\index{valor\_total\_time() (in module solucao\_fifa)@\spxentry{valor\_total\_time()}\spxextra{in module solucao\_fifa}}

\begin{fulllineitems}
\phantomsection\label{\detokenize{solucao_fifa:solucao_fifa.valor_total_time}}\pysiglinewithargsret{\sphinxcode{\sphinxupquote{solucao\_fifa.}}\sphinxbfcode{\sphinxupquote{valor\_total\_time}}}{\emph{\DUrole{n}{funcao\_atual\_ou\_futuro}}}{}
Função que retorna o valor total do time.
\begin{quote}\begin{description}
\item[{Parameters}] \leavevmode
\sphinxstyleliteralstrong{\sphinxupquote{funcao\_atual\_ou\_futuro}} (\sphinxstyleliteralemphasis{\sphinxupquote{function}}) \textendash{} Função que retorna os jogadores para calcular o valor da soma dos valores individuais de cada um.

\item[{Returns}] \leavevmode
Preço total do time.

\item[{Return type}] \leavevmode
float

\end{description}\end{quote}

\end{fulllineitems}



\chapter{Indices and tables}
\label{\detokenize{index:indices-and-tables}}\begin{itemize}
\item {} 
\DUrole{xref,std,std-ref}{genindex}

\item {} 
\DUrole{xref,std,std-ref}{modindex}

\item {} 
\DUrole{xref,std,std-ref}{search}

\end{itemize}


\renewcommand{\indexname}{Python Module Index}
\begin{sphinxtheindex}
\let\bigletter\sphinxstyleindexlettergroup
\bigletter{c}
\item\relax\sphinxstyleindexentry{classe\_Fifa\_limp}\sphinxstyleindexpageref{classe_Fifa_limp:\detokenize{module-classe_Fifa_limp}}
\item\relax\sphinxstyleindexentry{classe\_RS\_limp}\sphinxstyleindexpageref{classe_RS_limp:\detokenize{module-classe_RS_limp}}
\indexspace
\bigletter{f}
\item\relax\sphinxstyleindexentry{funcoes\_auxiliares}\sphinxstyleindexpageref{funcoes_auxiliares:\detokenize{module-funcoes_auxiliares}}
\indexspace
\bigletter{s}
\item\relax\sphinxstyleindexentry{solucao\_fifa}\sphinxstyleindexpageref{solucao_fifa:\detokenize{module-solucao_fifa}}
\end{sphinxtheindex}

\renewcommand{\indexname}{Index}
\printindex
\end{document}